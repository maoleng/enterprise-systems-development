% công nghệ dự kiến sẽ sử dụng 
% lý do lựa chọn công nghệ đó
Implementation is an important part to manage an application of a planning process. It improves to develop the strategic systems which must expect to include a process for applying the plan. After implementation a plan we need to develop the process according to the system.\\
Expected technologies to be used for a coffee website:\\
• Front-end: HTML, CSS, JavaScript, jQuery, Boostrap \\
• Back-end: PHP, Laravel, MySQL \\
• Deployment: Linux server
\section{Front-end}
HTML, CSS, JavaScript, jQuery, and Bootstrap for the front-end can be highly advantageous. HTML provides the foundation for creating the structure and content of a website, while CSS allows for the styling and presentation of that content. JavaScript and jQuery can then be used to add dynamic functionality and interactivity to the page, such as pop-up menus or form validation. Additionally, Bootstrap provides pre-built components and templates, making it easy to create a responsive design that looks great on all devices without spending a lot of time on design and layout. By using these tools together, developers can create a functional and visually appealing front end quickly, without having to reinvent the wheel. This can be especially beneficial when working on a fast project with a tight deadline, allowing developers to focus on other areas of the project and deliver a quality end product on time.
\section{Back-end}
The main reason to choose PHP and Laravel with MySQL for the back-end of a web application is because they offer a robust and scalable solution for building modern web applications quickly and efficiently. PHP is a widely used programming language with a vast community and extensive documentation, making it easy for developers to get started quickly. Laravel is a popular PHP framework that simplifies the development process by providing a set of pre-built tools and features, allowing developers to focus on creating high-quality code. MySQL is a reliable and scalable database management system that can handle large amounts of data, ensuring data consistency and availability. By using these technologies together, developers can create powerful, flexible, and secure web applications that can scale as the business grows, meeting the needs of its users and stakeholders.
\section{Deployment}
Reason for choosing Ubuntu and Apache to deploy a web application is that they offer a reliable, stable, and secure solution. Ubuntu is a popular operating system that is known for its stability, reliability, and security, making it a reliable choice for web application deployment. Apache is a widely-used and highly customizable web server software that provides excellent performance and security features. By using these technologies together, developers can ensure that their web application is running on a stable and secure platform, allowing them to focus on developing high-quality code and delivering a great user experience. Ubuntu and Apache have extensive documentation and a large community of users, making it easy to find solutions to problems and stay up-to-date with the latest developments in web application deployment.

